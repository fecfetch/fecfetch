\typeout{^^J ict_beamer.tex, version 0.1,  2016-03-07^^J}


%--------------------------------------------------------------------------------------------------
% TODO:
%	 o Kopfzeile fehlt komplett wenn frame durch \begin{frame} und nicht durch \begin{frame}{}{} erzeugt wird
%   o Seltsames Verhalten der minipages in den Fußzeilen. Evtl. durch parboxen ersetzen und Höhen anpassen
%   o \usecolortheme, \useinnertheme loswerden und einfach alles selbst definieren
%   o Make a nice color pallette
%   o \reserveandshow erweitern für beliebige Seitenzahlen (wirklich?)
%   o \islanguage oder wie das heißt für Sprache einbauen
%   o \soundexample durch \dsssound ersetzt, das auf media9 basiert (Nachfolger von movie15). Kann optisch
%     noch ein wenig verbessert werden. Miktex 2.9 nötig! Wenn kein Text für den Player-Icon angegeben und LaTeX->PS->...
%     kompiliert wird, muss dvips ohne die Option -p pdf aufgerufen werden!
%
% Irgendwann mal
%		o Das alles hier in .sty verwandeln
%
%--------------------------------------------------------------------------------------------------


%--------------------------------------------------------------------------------------------------
% Packages
%--------------------------------------------------------------------------------------------------
\usepackage{graphicx}
\usepackage{amsmath}
\usepackage{amsfonts}
\usepackage{amssymb}
\usepackage[utf8]{inputenc}
\usepackage[T1]{fontenc}
\usepackage{lmodern}
\usepackage{verbatim}
\usepackage{pgfpages}
\usepackage{booktabs}
\usepackage{overpic}
\usepackage{psfrag}
\usepackage[bigfiles]{media9}
\usepackage{multimedia}
\usepackage{hyperref}
\usetikzlibrary{mindmap,trees}


%--------------------------------------------------------------------------------------------------
% Define macros other
%--------------------------------------------------------------------------------------------------
\newcommand{\headline}[1]{\textcolor{tf_blue_light}{\textbf{#1}}}
\newcommand{\highlight}[1]{\textcolor{tf_blue_light}{#1}}

% --- Show a picture only from the second slide on. For the first slide an empty box of the same
%     size as the graphic is created to avoid wobbeling of the content
\newcommand{\reserveandshow}[2][]{
    \phantom{\includegraphics<+>[#1]{#2}}
    \includegraphics<+->[#1]{#2}
}


% --- Include audio files: using the new media9 package. Has a nice status bar during playback.
% Use as \dsssound{<file>}{<text for the box>}{<boxcolor>}{<boxwidth>}
% Remember that <file> canNOT include the path. Set the media path in the TeX document's
% preamble using \addmediapath{<directory>}
\newcommand{\dsssound}[4]{{%Note: needs the double {{ for use in tabulars (http://tex.stackexchange.com/questions/147744/media9-in-tables)
    \includemedia[
        addresource=#1,
        transparent,
        deactivate=onclick,
        flashvars={
            source=#1
            &autoPlay=true
        },
    ]{\begin{beamercolorbox}[rounded=true, wd=#4, shadow=false]{#3}
        #2
      \end{beamercolorbox}}{APlayer.swf}%
}}

% --- Block environment with variable width, taken from http://www.latex-community.org/viewtopic.php?f=4&t=2251
\newenvironment<>{varblock}[2][\textwidth]{
      \begin{minipage}{#1}
        \setlength{\textwidth}{#1}
          \begin{actionenv}#3
            \def\insertblocktitle{#2}
            \par
            \usebeamertemplate{block begin}}
  {\par
      \usebeamertemplate{block end}
    \end{actionenv}
  \end{minipage}}

%--------------------------------------------------------------------------------------------------
% Beamer themes
%--------------------------------------------------------------------------------------------------
%==== beamer themes ====
\usetheme{default}                 % presentation theme
\usepackage{./sty/beamerouterthemeict}  % serving as outer theme 
\useinnertheme[shadow]{rounded}    % inner theme
\usecolortheme{orchid}             % inner colorscheme
\usefonttheme[onlymath]{serif}     % use serif fonts for mathematical text

\usefonttheme[onlymath]{serif}     % Use serif fonts for mathematical text

\setbeamertemplate{headline}[default]

\setbeamersize{text margin left = 0.5cm, text margin right = 0.5cm} % sets frame margins
\beamertemplatenavigationsymbolsempty % no navigation symbols
\setbeamercovered{invisible} % uncovered objects are invisible, other are "dynamic", "higly dynamic", "tranparent=n%"

%--------------------------------------------------------------------------------------------------
% Define colors
%--------------------------------------------------------------------------------------------------
% --- Original CAU colors for TF
%\definecolor{tf_blue_light}{RGB}{0, 43, 217}   % Fakultätsfarbe: Pantone 280
\definecolor{tf_blue_dark}{RGB}{0, 0, 191}     % Mischfarbe: HKS 33 + Pantone 280

% --- Colors of ppt template
\definecolor{tf_blue_light}{RGB}{0, 33, 146}  % color from PPT-template
\definecolor{dss_light_gray}{RGB}{216, 216, 216}  % color from PPT-template
\definecolor{dss_dark_gray}{RGB}{149, 149, 149}  % color from PPT-template

% --- My color pallette
\definecolor{jow_yellow}{RGB}{247, 212, 143}
\definecolor{jow_green}{RGB}{178, 212, 178}
\definecolor{jow_blue}{RGB}{117, 166, 214}
\definecolor{jow_red}{RGB}{241, 167, 144}

% --- Make all colors available as beamer colors
\setbeamercolor{tf_blue_dark}{bg=tf_blue_dark}
\setbeamercolor{tf_blue_light}{bg=tf_blue_light}
\setbeamercolor{dss_light_gray}{bg=dss_light_gray}
\setbeamercolor{dss_dark_gray}{bg=dss_dark_gray}

\setbeamercolor{jow_green}{bg=jow_green}
\setbeamercolor{jow_blue}{bg=jow_blue}
\setbeamercolor{jow_yellow}{bg=jow_yellow}
\setbeamercolor{jow_red}{bg=jow_red}


%--------------------------------------------------------------------------------------------------
% Set colors
%--------------------------------------------------------------------------------------------------
%--- Colors of header
\setbeamercolor{frametitle}{fg=white,bg=tf_blue_light} 
\setbeamercolor{framesubtitle}{fg=white,bg=tf_blue_light} % old: 'fg=black,bg=dss_light_gray'; new: fg=white,bg=tf_blue_light
% \setbeamercolor{framesubtitle}{fg=black,bg=dss_light_gray}

\setbeamercolor{title}{fg=black, bg=dss_dark_gray}
\setbeamerfont{institute}{size = \scriptsize}
\setbeamerfont{date}{parent=institute}
\setbeamerfont{title}{parent=institute} 
\setbeamerfont{subtitle}{parent=institute} 

% --- Colors of blocks
\setbeamertemplate{blocks}[rounded][shadow=true]
\setbeamercolor{block title}{fg=white,bg=tf_blue_light}
\setbeamercolor{block body}{bg=dss_light_gray}
\setbeamercolor{structure}{fg=tf_blue_light}

%--------------------------------------------------------------------------------------------------
% Set bullets for...
%--------------------------------------------------------------------------------------------------
% --- Table of contents
\setbeamertemplate{sections/subsections in toc}[sections numbered]
\setbeamertemplate{subsection in toc}{\qquad -  \inserttocsubsection \par }

% --- Itemizations
\setbeamertemplate{itemize item}[triangle]
\setbeamertemplate{itemize subitem}[circle]
\setbeamertemplate{itemize subsubitem}[circle]

% --- Enumerations
\setbeamertemplate{enumerate items}[default]

%--------------------------------------------------------------------------------------------------
% Set titleframe
%--------------------------------------------------------------------------------------------------
\setbeamertemplate{title page}
{
    %------------------------------------------------------------------------------------------------
    % Blue box
    %------------------------------------------------------------------------------------------------
    \vspace{-2.1mm}
    %nj: default for 4:3: [wd=\paperwidth,ht=7.4mm,leftskip=2mm,rightskip=-1.5mm,dp=1.5mm]
    \begin{beamercolorbox}[wd=\paperwidth,ht=7mm,leftskip=2mm,rightskip=-1.5mm,dp=4.5mm]{frametitle}
        \hfill
        %\begin{minipage}{0.235\paperwidth}
        \begin{minipage}{0.2\paperwidth}
            %\vspace{-4.25mm}
            \vspace{1mm}
            \includegraphics[width=\textwidth]{\StyleDir cau-sw}
        \end{minipage}
    \end{beamercolorbox}

    \vskip-0.4mm % to remove a tiny white space between both bars

    %------------------------------------------------------------------------------------------------
    % Dark gray box
    %------------------------------------------------------------------------------------------------
    % default for 4:3: ht = 10mm, dp = 10mm
    \begin{beamercolorbox}[wd=\paperwidth,ht=6mm,leftskip=2mm,dp=6mm, rightskip = 0.5mm]{frametitle} % changed 'title' to frametitle [nd]
        \begin{minipage}[c]{1.0\paperwidth}
            \usebeamerfont{title} \inserttitle \newline \usebeamerfont{subtitle} \insertsubtitle
        \end{minipage}
    \end{beamercolorbox}

    %------------------------------------------------------------------------------------------------
    % Light gray box
    %------------------------------------------------------------------------------------------------
    \vskip-1.4mm % to remove a tiny white space between both bars
    \begin{beamercolorbox}[wd=\paperwidth,ht=62.0mm,leftskip=2mm,dp=8mm, rightskip = 0.5mm]{white} % changed dss_light_grey to white [nd]
        \begin{minipage}[b]{0.5\paperwidth}
            \insertauthor \newline

            \usebeamerfont{institute}     \vspace{5mm}
            \insertdate \newline

            \insertinstitute
        \end{minipage}
        %----------------------------------------------------------------------------------------------
        % CAU-logo (seal)
        %----------------------------------------------------------------------------------------------
        \begin{minipage}[c]{0.5\paperwidth}
            \begin{center}
                \includegraphics[width=1.3\textwidth]{\StyleDir cau_seal} % changed width from 0.8 to 1.3 [nd]
            \end{center}
        \end{minipage}
    \end{beamercolorbox}
}

%--------------------------------------------------------------------------------------------------
% Set frametitle
%--------------------------------------------------------------------------------------------------
\setbeamertemplate{frametitle}
{
    %------------------------------------------------------------------------------------------------
    % Blue box
    %------------------------------------------------------------------------------------------------
  	%%% Default for 4:3:
    %\vspace{-1.75mm}
    %\begin{beamercolorbox}[wd=\paperwidth,ht=7.4mm,leftskip=2mm,rightskip=-1.5mm,dp=2.75mm]{frametitle}
    %    \usebeamerfont{frametitle} \insertframetitle 
    %   \hfill
    %    \begin{minipage}[c]{0.235\paperwidth}
    %        \vspace{-1mm}
    %        \includegraphics[width=\textwidth]{\StyleDir cau-sw}
    %    \end{minipage}
    %\end{beamercolorbox}
	%%% nj: settings for 16:9:
	\vspace{-0.75mm}
	\begin{beamercolorbox}[wd=\paperwidth,ht=7mm,leftskip=2mm,rightskip=-1.5mm,dp=2.75mm]{frametitle}
		\usebeamerfont{frametitle} \insertframetitle
		\hfill
		\begin{minipage}[c]{0.2\paperwidth}
			\vspace{-1mm}
			\includegraphics[width=\textwidth]{\StyleDir cau-sw}
		\end{minipage}
	\end{beamercolorbox}
    \vskip-0.6mm % to remove a tiny white space between both bars

    %------------------------------------------------------------------------------------------------
    % Light gray box
    %------------------------------------------------------------------------------------------------
    \if\insertframesubtitle\empty
    \else
        \begin{beamercolorbox}[wd=\paperwidth,ht=3mm,leftskip=2.5mm,dp=1mm, rightskip = 0.5mm]{framesubtitle}
            \usebeamerfont{framesubtitle} \insertframesubtitle
        \end{beamercolorbox}
    \fi
}

%--------------------------------------------------------------------------------------------------
% Set footline
%--------------------------------------------------------------------------------------------------
\setbeamertemplate{footline}{

\hrule height 0.3pt % this is new [nd]

%------------------------------------------------------------------------------------------------
    % Left section: DSS logo
    %------------------------------------------------------------------------------------------------
    \begin{beamercolorbox}[wd=\paperwidth,ht=1.0mm,leftskip=3mm,dp=3.5mm]{white} % changed 'framesubtitle' to 'white' [nd]
        \begin{minipage}[t][0.25cm][c]{0.95cm}
            \includegraphics[width=\textwidth]{\StyleDir ict-logo-white.pdf} % for grey background ict-logo-grey.pdf is needed [nd]
        \end{minipage}
        %
        \begin{minipage}[t][0.25cm][c]{17mm}
            \bfseries
            \scalebox{0.9}{Chair of Information} \\
            \scalebox{0.9}{and Coding Theory}
        \end{minipage}
        %
        \hspace{3mm}
        %
        %----------------------------------------------------------------------------------------------
        % Middle section: author, title
        %----------------------------------------------------------------------------------------------
        \begin{minipage}[t][0.25cm][c]{60mm}
            % Don't ask me why the scalebox is needed. Anyway, it prevents from a linebreak after \insertshortauthor
           % \scalebox{1.0}{\insertshortauthor \;$|$ \insertshorttitle}
            \scalebox{1.0}{\insertshortauthor \;$|$ \insertshorttitle}
        \end{minipage}
        %
        \hfill
        %----------------------------------------------------------------------------------------------
        % Right section: slide number
        %----------------------------------------------------------------------------------------------
        \begin{minipage}[t][0.25cm][c]{15mm}
            \SlideName \; {\insertframenumber}/{\inserttotalframenumber}
        \end{minipage}
    \end{beamercolorbox}
}

%--------------------------------------------------------------------------------------------------
% Command for packing four slides on one DIN A4 page (works only with pdflatex!)
%--------------------------------------------------------------------------------------------------
\newcommand{\makeHandout}{
    \pgfpagesuselayout{4 on 1}[a4paper,border shrink=5mm, landscape]
    \mode<handout>{
%          \setbeamercolor{background canvas}{bg=black!5}
        % Make a border around frames
        \setbeamercolor{background canvas}{ \tikz \draw[black]
            (current page.north west)
            rectangle
            (current page.south east);
        }
    }
}


%--------------------------------------------------------------------------------------------------
% Keys to support piece-wise uncovering of elements in TikZ pictures:
%     \node[visible on=<2->](foo){Foo}
%
% Helpful, e.g., for including overlays in TikZ mindmaps.
% Found here: http://tex.stackexchange.com/questions/55806/tikzpicture-in-beamer
%--------------------------------------------------------------------------------------------------
\tikzset{
    invisible/.style={opacity=0},
    visible on/.style={alt=#1{}{invisible}},
    alt/.code args={<#1>#2#3}{%
        \alt<#1>{\pgfkeysalso{#2}}{\pgfkeysalso{#3}}
    },
}
