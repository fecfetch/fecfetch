\documentclass[10pt,aspectratio=169]{beamer}


%--------------------------------------------------------------------------------------------------
% Set language options and load ICT beamer template
%    Set the directory of the template by \newcommand{\StyleDir}{<yourDir>}. <yourDir> can be a
%    relative or absolute path and is necessary to find the images contained in the template.
%--------------------------------------------------------------------------------------------------
\newcommand{\SlideName}{Slide}
\newcommand{\StyleDir}{sty/}
\usepackage{tikz}
\input{sty/ict_beamer169.tex}
%% papersize is 128mm times 96mm.
% Last change: 2007-07-06, jf 
% 2007-07-06, Justus Fricke
%  - ./style/beamerouterthemeict.sty changed for pgf >= 1.00
%  - version output added
%  - new IEEEtran.bst (version 1.12) included in ./style
%  - changed figure environment (caption) [chk] 23.03.2011

\typeout{^^J-- beamer_ict.tex, version 1.1,  2007-07-06^^J}

%==== packages ====
\usepackage{verbatim}
\usepackage{graphicx}
\usepackage{overpic}

%==== beamer themes ====
\usetheme{default}                 % presentation theme
\usepackage{./style/beamerouterthemeict}  % serving as outer theme 
\useinnertheme[shadow]{rounded}    % inner theme
\usecolortheme{orchid}             % inner colorscheme
\usefonttheme[onlymath]{serif}     % use serif fonts for mathematical text


%===========================================================
% Template settings
%===========================================================

\definecolor{tf_blue}{RGB}{10,5,74}     % TF blue
\definecolor{tf_red}{RGB}{161,0,32}     % TF red
\definecolor{ptf_blue}{RGB}{10,5,124}   % pseudo (light) TF blue

\setbeamertemplate{headline}[default]   % default: leave headline empty


% override the beamer default footline
\setbeamertemplate{footline}{
  \hrule
  \vspace{0.2cm}
  \begin{minipage}[t][0.25cm][c]{1.3cm}
     %\hspace{0.1cm}
     \includegraphics[scale=0.21]{./style/logo_hensoldt.jpg}
     \vspace{0.02cm}
  \end{minipage}
  \hspace{6.0cm}
  \begin{minipage}[t][0.25cm][c]{3cm}
%    Faculty of Engineering\\
%    University of Kiel
  \end{minipage}
  %\hfill
  \begin{minipage}[t][0.25cm][c]{1cm}
    \begin{center}
      \insertframenumber\\    % to display the current frame number in the footline      
    \end{center}
  \end{minipage}
  \hspace{2.2cm}
%   \begin{minipage}[t][0.25cm][c]{1.8cm}
%  	\begin{flushleft}
%  		Maurice Hott\\
%  		maho@tf.uni-kiel.de
%  	\end{flushleft}
%  \end{minipage}
  %\hspace*{0.5cm}
  %\hfill
%  \begin{minipage}[t][0.28cm][c]{1.6cm}
%    \begin{flushright}
%      Chair of Information\\
%      and Coding Theory
%    \end{flushright}
%  \end{minipage}
  \begin{minipage}[t][0.28cm][c]{3.4cm}
	\begin{flushright}
		Maurice Hott $\vert$ maho@tf.uni-kiel.de\\
		Kiel University
	\end{flushright}
\end{minipage}
  \hspace{0.1cm}
  \begin{minipage}[t][0.25cm][c]{1.2cm}
     \includegraphics[scale=0.105]{./pics/cau-logo-color.png}
  \end{minipage}
  \vspace{0.16cm}
}


\setbeamersize{text margin left = 0.5cm, text margin right = 0.5cm} % sets frame margins

\setbeamercolor{frametitle}{fg=white,bg=ptf_blue!70!blue!}          % define start colour for frame title shading
\setbeamercolor{frametitle right}{fg=white, bg=tf_blue}    	    % define end colour for frame title shading
\setbeamercolor{block title}{bg=ptf_blue}
\setbeamercolor{structure}{fg=ptf_blue}

\setbeamercolor{title}{fg=white,bg=ptf_blue}	% block around the title
\setbeamerfont{date}{parent=institute}          % use the same font for the date as for the institute entry

\setbeamercovered{invisible} % uncovered objects are invisible, other are "dynamic", "higly dynamic", "tranparent=n%"

\beamertemplatenavigationsymbolsempty % no navigation symbols


% set bullets for itemizations
\setbeamertemplate{itemize item}[triangle]
\setbeamertemplate{itemize subitem}[circle]
\setbeamertemplate{itemize subsubitem}[circle]

% set bullets for enumerations
\setbeamertemplate{enumerate items}[default]

% set bullets for table of contents
\setbeamertemplate{sections/subsections in toc}[sections numbered]

\logo{} % the position of this logo depends on the used outer theme

\usepackage{sty/ictcommands}
\usepackage{sty/abbr}
%\usepackage{import}
\usepackage{pgfplots}

\usepackage{verbatim}
\usetikzlibrary{shapes, arrows,graphs}

\graphicspath{{./pic/}}
\newcommand{\minipagegraphwidth}{0.65\columnwidth}
\newcommand{\graphwidth}{0.85\columnwidth}
\newcommand{\minipagebulletwidth}{0.34\columnwidth}
\DeclareMathOperator{\F}{\mathcal{F}}
%\pgfplotsset{compat=1.5}
%--------------------------------------------------------------------------------------------------
% Pack four slides on one DIN A4 page
%--------------------------------------------------------------------------------------------------
% \makeHandout


%--------------------------------------------------------------------------------------------------
% Title settings, e.g. author, title, ... 
%--------------------------------------------------------------------------------------------------
\title{Polarization in Communication Channel Models}
\subtitle{Comparison of SCM with cross-polarized and single-polarized antenna arrays utilizing Matlab simulation }
\author[F.E.~Çorumluoğlu]{Fatih\,E.\,Çorumluoğlu}

\date[09.02.2021]{Kiel, February 21$^{\textit{st}}$ , 2021}
\institute {Kiel University} 
            
            
%--------------------------------------------------------------------------------------------------
% Environment for no vertical spacing of two plots
%--------------------------------------------------------------------------------------------------
%\newenvironment{pics}
%{\par\raggedright % maybe \centering
%	\setlength\tabcolsep{0pt}\renewcommand{\arraystretch}{0.1}%
%	\begin{tabular}{*{10}c}}
%	{\end{tabular}\par}

\begin{document}
	
	\setcounter{tocdepth}{1}
	\setbeamercovered{dynamic}
	\frame[plain]{\titlepage      % titlepage without headline, footline and frametitle
		\setcounter{framenumber}{0}}  % reset pagenumber (firstpage will be No. 1)
	
	% %--------------------------------------------------------------------------------------------------
	% % Table of contents
	% %--------------------------------------------------------------------------------------------------
	% \section*{Content}
	% % --- Give this slide the name 'contentframe' so it can be repeated later later on
	% \begin{frame}<1>[label = contentframe]{Content}{\quad}
	%   \setcounter{tocdepth}{1}
	%   \tableofcontents
	% \end{frame}
	
	%--------------------------------------------------------------------------------------------------
	% Include sections
	%--------------------------------------------------------------------------------------------------
	
	%================ New frame ================================
	\begin{frame}
		\frametitle{Outline}
		\setcounter{tocdepth}{1}
		\tableofcontents
	\end{frame}
	
	\section{Introduction}
	%================ New frame ================================
	\begin{frame}
		\frametitle{\insertsection}
		\framesubtitle{\insertsubsection}
		Basic descriptions
        
\begin{itemize}
\item Communication channels
\end{itemize}
\begin{itemize}
\item Communication channel models

\end{itemize}
	\end{frame}
	\section{Communication Channel Models}
	%================ New frame ================================
	\begin{frame}
		\frametitle{\insertsection}
		\framesubtitle{\insertsubsection}
		The 3 main communication channel model
        
\begin{enumerate}
\item SCM (Spatial Channel Model)
\item SCM-E (Spatial Channel Model Extended)
\item WINNER and WINNER II
\end{enumerate}
	\end{frame}
	
	\section{Polarization}
	%================ New frame ================================
	
	\begin{frame}
		\frametitle{\insertsection}
		\framesubtitle{\insertsubsection}
	
\begin{itemize}
\item What it is
\item Why it is important
\item Where is it used
\end{itemize}

		
	\end{frame}
	
    \section{Software Implementation}
    \subsection{Objective}
	%================ New frame ================================
	\begin{frame}
		\frametitle{\insertsection}
		\framesubtitle{\insertsubsection}
		Simulating a channel model to show and compare the effects of polarization 
		\[\includegraphics[scale=0.47 ]{polar.png}\]
	\end{frame}
	
	\subsection{Choosing the right properties}
	%================ New frame ================================
	\begin{frame}
		\frametitle{\insertsection}
		\framesubtitle{\insertsubsection}
\begin{itemize}
\item Channel Model
\item Environment
\end{itemize}
	\end{frame}
	\subsection{Initialization}
	%================ New frame ================================
	\begin{frame}
		\frametitle{\insertsection}
		\framesubtitle{\insertsubsection}
	  Setting the initial parameters
	  \[\includegraphics[scale=0.42 ]{tablo.png}\]
	\end{frame}
		\subsection{Antenna setup}
	%================ New frame ================================
	\begin{frame}
		\frametitle{\insertsection}
		\framesubtitle{\insertsubsection}	 
        Antenna setup process:
\begin{enumerate}
\item Define a  reference position and locate the first antenna in that position. 
\item Locate the next antennas according to the distance setting
\item Repeat for BS and MS
\end{enumerate}

	\end{frame}
	\subsection{Environment setup}
	%================ New frame ================================
	\begin{frame}
		\frametitle{\insertsection}
		\framesubtitle{\insertsubsection}	 
        Environment setup
\begin{enumerate}
\item Define the additional environment parameters
\item Define Path-loss function
\end{enumerate}

	\end{frame}
	\subsection{Power delay setup}
	%================ New frame ================================
	\begin{frame}
		\frametitle{\insertsection}
		\framesubtitle{\insertsubsection}	 
        Power delay setup
\begin{enumerate}
\item Random delays
\item Sorting
\item Power computation
\item Normalize the power
\end{enumerate}

	\end{frame}
		\subsection{AoD and AoA setup}
		%================ New frame ================================
	\begin{frame}
		\frametitle{\insertsection}
		\framesubtitle{\insertsubsection}	 
        AoD and AoA setup
\begin{enumerate}
\item Random initial values
\item Sorting
\item Calculation of angle of departure values
\item Calculation of angle of arrival values
\end{enumerate}

	\end{frame}
	\subsection{Channel Matrices}
		%================ New frame ================================
	\begin{frame}
		\frametitle{\insertsection}
		\framesubtitle{\insertsubsection}	 
        Channel matrices needs to be set up separately according to polarization
        \begin{itemize}
			\item<2->  Single-polarized
			\begin{itemize}
			\item Random phase, AoD and AoA
			\item Antenna gain
			\item Channel Impulse response matrix
            \end{itemize}

			\item<3->  Cross-Polarized
			\begin{itemize}
			\item Random phase, AoD and AoA
			\item Generate additional antennas for BS and MS
			\item Channel Impulse response matrix
            \end{itemize}
			
		\end{itemize}

	\end{frame}
	\subsection{Capacity}
		%================ New frame ================================
	\begin{frame}
		\frametitle{\insertsection}
		\framesubtitle{\insertsubsection}	 
        Capacity also needs to be calculated separately according to polarization
\[C=log_2 det(I+P/\sigma^2*H*H^H)\]
\[P/\sigma^2=SNR/S\]
	\end{frame}
	\section{Results}
	\subsection{4 BS and MS antennas}
		%================ New frame ================================
	\begin{frame}
		\frametitle{\insertsection}
		\framesubtitle{\insertsubsection}
		\begin{figure}[h!]
  \includegraphics[scale=0.30 ]{1.PNG}
  \caption{S,U=4;DBS=[6,6,6],DMS=[0.4,0.4,0.4];aBS,bMS=0;R=1700;T=100;SNR=15}
\end{figure}
\end{frame}
	\subsection{Radius is smaller}
		%================ New frame ================================
	\begin{frame}
		\frametitle{\insertsection}
		\framesubtitle{\insertsubsection}	 
       \begin{figure}[h!]
  \includegraphics[scale=0.30]{yarıçapkısa4.PNG}
  \caption{S,U=2;DBS=[6],DMS=[0.4];aBS,bMS=0;R=1000;T=100;SNR=15}
\end{figure}

	\end{frame}
	\subsection{Time frame is greater}
		%================ New frame ================================
	\begin{frame}
		\frametitle{\insertsection}
		\framesubtitle{\insertsubsection}	 
       \begin{figure}[h!]
  \includegraphics[scale=0.30]{tbüyük3.PNG}
  \caption{S,U=2;DBS=[6],DMS=[0.4];aBS,bMS=0;R=1700;T=1000;SNR=15}
\end{figure}

	\end{frame}
        
       
	\section{Conclusion}
		%================ New frame ================================
	\begin{frame}
		\frametitle{\insertsection}
		\framesubtitle{\insertsubsection}	 
        \begin{itemize}
        \item What we have seen
        \item What it means
        \end{itemize}
	\end{frame}
	\appendix
	
	%--------------------------------------------------------------------------------------------------
	% End of document
	%--------------------------------------------------------------------------------------------------
\end{document}
